\documentclass{article}
\usepackage[utf8]{inputenc}

\title{CS 6000 Journal 5}
\author{David Stout}
\date{September 2018}

\begin{document}

\maketitle

\section{Learning how to write a survey paper}
I have really struggled with trying to understand what a Survey paper is. I believe that I was on the right path with my rough draft, but to be sure I am having my PhD advisor review my rough draft and help me to refine my process a little more. 

Crafting the story has been a struggle, I am finding it tough to make such a technical thing into a interesting story for any person. I kept attempting to just produce a report of the current technologies, but I know that is not solely what this is about. I have been speaking with other students and it seems like a lot of people have struggled with this. Upon the advice of Dr. Boult, I have been just writing as much as I can without deleting. Even though it may be disconnected right now, I hope to have enough stubs to be able to craft a bit of the story of each one. I currently have a whole bunch of disconnected thoughts, and am trying to make sure that there is a way that I connect them as much as possible.

I have had my wife and friends go over what I have written so far, in the hopes of getting some feedback from sources that are not in the Computer Science field so that I can write something that is able to be read by everyone. I am also talking with other CS students to get some more professional feedback. I hope that all of this feedback will help me to refine my writing and assist with the editing of the paper. 

Moving forward, I will be working towards making a cohesive and interesting story. I will also be adjusting my calendar to make sure that all the deadlines are up to date, so that I don't miss anymore due dates. I am not sure how I got so mixed up for the due dates, but I need to make sure that it is not an issue going forward. I want to do the best that I can for this project and semester, so I have to be diligent and make sure that my time management is no longer an issue. 


\end{document}
